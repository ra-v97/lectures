\input{include/header}


\usepackage{tikz}
\usetikzlibrary{decorations.pathreplacing,calc}

\newcommand{\tikzmark}[1]{\tikz[overlay,remember picture] \node (#1) {};}

\newcommand*{\AddNote}[5]{
    \begin{tikzpicture}[overlay, remember picture]
        \draw [decoration={brace,amplitude=0.5em},decorate,ultra thick,red]
            ($(#3)!(#1.north)!($(#3)-(0,1)$)$) --
            ($(#3)!(#2.south)!($(#3)-(0,1)$)$)
                node [align=center, text width={#4}, pos=0.5, anchor=west] {#5};
    \end{tikzpicture}
}

%%%%%%%%

\subtitle{10. Numeryczna algebra liniowa -- wprowadzenie.}
\setcontributors{Magdalena Nowak\\Paweł Taborowski\\Rafał Stachura}


\begin{document}
  \maketitle
	\begin{frame}{Plan wykładu}
		\tableofcontents
	\end{frame}

  \input{10_numeryczna_algebra_liniowa-wprowadzenie/10_1_zastosowanie_num_algebry_liniowej+10_2_macierze_tekstowe}

  \section{Dlaczego algorytmy równoległe?}

\begin{frame}{Dlaczego algorytmy równoległe?}

\begin{exampleblock}{Przykład -- przewidywanie pogody (globalnie):}
rozwiązywanie równania Naviera-Stokesa na siatce 3D wokół Ziemi

Zmienne:

  $\left. \parbox{10em}
{\begin{itemize}
   \item \text{temperatura}
   \item \text{ciśnienie}
   \item \text{wilgotność}
   \item \text{prędkość wiatru}
  \end{itemize}}
\right \} \text{równanie Naviera-Stokesa (6 zmiennych)}$
\end{exampleblock}

Obliczenia:

$\left. \parbox{15em}
{\begin{itemize}
   \item \text{elementarna komórka 1 km}
   \item \text{10 warstw}
  \end{itemize}}
\right \} 5\cdot10^9\text{ komórek}$

\begin{itemize}
    \item w każdej komórce:
	$6\cdot 8 \text{ Bytes} \Rightarrow 2\cdot 10^{11} \text{ Bytes} = 200 \text{ Gbytes}$,

    \item w każdej komórce 100 operacji fp,
    \item obliczenia 1 kroku czasowego 1 min $\Rightarrow \frac{100\cdot5\cdot10^9}{60} = 8 \text{ GFLOPS}$
 \end{itemize}

\end{frame}


  \section{Basic matrix algorithms}

  \input{10_numeryczna_algebra_liniowa-wprowadzenie/10_4_1_dot_product}

  \input{10_numeryczna_algebra_liniowa-wprowadzenie/10_4_4_matrix-vector_multiplication}

  \input{10_numeryczna_algebra_liniowa-wprowadzenie/10_4_5_the_gaxpy_computation+10_4_6_outer}

 \input{10_numeryczna_algebra_liniowa-wprowadzenie/10_4_7_matrix_multiplication}
 
\section{Biblioteki}

\begin{frame}[fragile]{Biblioteki dla algebry liniowej}

\begin{itemize}
\item BLAS - Basic Linear Algebra Subprograms
	\begin{itemize}
		\item \url{http://www.netlib.org/blas/}
		\item \url{https://www.gnu.org/software/gsl/doc/html/blas.html?highlight=blas#c.gsl_blas_ddot}
	\end{itemize}

\item  LAPACK - Linear Algebra Package
	\begin{itemize}
		\item \url{http://www.netlib.org/lapack/}
	\end{itemize}
\item JAMA - Java Matrix Package
	\begin{itemize}
		\item \url{https://math.nist.gov/javanumerics/jama/}
	\end{itemize}
\item GNU Octave
	\begin{itemize}
		\item \url{https://www.gnu.org/software/octave/about.html}
	\end{itemize}
\end{itemize}
$\newline$
\textbf{Uwaga:} Więcej o bibliotekach numerycznych w prezentacji 13.

\end{frame}

\section{Bibliografia}
\begin{frame}{Bibliografia}
  \begin{thebibliography}{9}
    \setbeamertemplate{bibliography item}[online]
      \bibitem[HARWELL-BOEING]{matrix}{Matrix Market \newblock The Harwell-Boeing Sparse Matrix Collection \newblock \small{\url{http://math.nist.gov/MatrixMarket/data/Harwell-Boeing/}}}
  \end{thebibliography}
\end{frame}

\end{document}
